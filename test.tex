\documentclass{article}

% Packages
\usepackage[T1]{fontenc}   % Font encoding
\usepackage[ngerman]{babel} % German language
\usepackage{csquotes} % For context-sensitive quotation marks
\usepackage[backend=biber, style=authoryear]{biblatex} % For bibliography

% Bibliography file
\addbibresource{Sources.bib}

\title{Minimalbeispiel für Zitate}
\author{Flo}
\date{\today}

\begin{document}

\maketitle

\section{Einführung}
Dies ist ein Minimalbeispiel, um das Zitieren zu testen. Hier ist ein Beispielzitat~\cite{Zweig.1964b}.

\section{Weitere Beispiele}
Ein weiteres Beispielzitat~\cite{GellMann.1964, Zweig.1964} zeigt, wie man mehrere Quellen zitiert.

Soo, jetzt versuche ich noch etwas ganz gewagtes zu zitieren:~\cite{C.Amsler.2017}.

Und wie siehts mit Onlinequellen aus?~\cite{Wikipedia.Standardmodell}

\printbibliography\end{document}