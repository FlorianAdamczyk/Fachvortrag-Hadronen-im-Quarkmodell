\documentclass[aspectratio=169]{beamer} % 16:9 aspect ratio for modern screens

% Theme settings
\usetheme{metropolis} % Minimalist theme
\usefonttheme{professionalfonts} % Font theme

% Packages
\usepackage[utf8]{inputenc} % Encoding
\usepackage[T1]{fontenc}   % Font encoding
\usepackage[ngerman]{babel} % German language
\usepackage{graphicx}       % For including images
\usepackage{amsmath, amssymb} % For math symbols
\usepackage{tikz}          % For creating diagrams
\usepackage{booktabs}      % For better tables
\usepackage{biblatex}      % For bibliography
\usepackage{csquotes}      % Recommended for biblatex with babel/polyglossia

% Bibliography file
\addbibresource{literatur.bib}

% Title page settings
\title{Hadronen im Quarkmodell}
\subtitle{Eine physikalische Betrachtung}
\author{Dein Name}
\date{Datum}

\begin{document}
	
	% Title Slide
	\begin{frame}
		\titlepage\end{frame}
	
	% Table of Contents
	\begin{frame}{Inhaltsverzeichnis}
		\tableofcontents
	\end{frame}
	
	% Section: Einführung
	\section{Einführung}
	\begin{frame}{Einführung}
		\frametitle{Einführung}
		\begin{block}{Ziel der Präsentation}
			\begin{itemize}
				\item Vorstellung der Grundlagen des Quarkmodells
				\item Beschreibung der Hadronenstruktur
				\item Diskussion aktueller Herausforderungen
			\end{itemize}
		\end{block}
		\begin{figure}
			\centering
			\includegraphics[width=0.6\linewidth]{example-image} % Replace with actual image path
			\caption{Darstellung eines Hadronenmodells}
		\end{figure}
		\vfill
		\tiny Quelle: Beispielquelle für das Bild
	\end{frame}
	
	% Section: Theoretische Grundlagen
	\section{Theoretische Grundlagen}
	\begin{frame}{Quarkmodell: Ein Überblick}
		\frametitle{Theoretische Grundlagen}
		\begin{itemize}
			\item Quarks als fundamentale Bausteine
			\item Drei Farbladungen: Rot, Grün, Blau
			\item Austausch von Gluonen $\Rightarrow$ starke Wechselwirkung
		\end{itemize}
		\begin{equation}
			F = \frac{Gm_1m_2}{r^2} % Beispiel für eine Formel
		\end{equation}
	\end{frame}
	
	\begin{frame}{Hadronenklassen}
		\frametitle{Theoretische Grundlagen}
		\begin{block}{Baryonen}
			\begin{itemize}
				\item Drei Quarks
				\item Beispiele: Proton, Neutron
			\end{itemize}
		\end{block}
		\begin{block}{Mesonen}
			\begin{itemize}
				\item Ein Quark und ein Antiquark
				\item Beispiele: Pionen, Kaonen
			\end{itemize}
		\end{block}
	\end{frame}
	
	% Section: Experimentelle Beobachtungen
	\section{Experimentelle Beobachtungen}
	\begin{frame}{Nachweis von Quarks}
		\frametitle{Experimentelle Beobachtungen}
		\begin{figure}
			\centering
			\includegraphics[width=0.7\linewidth]{example-image} % Replace with actual image path
			\caption{Streuexperiment zur Quarkstruktur}
		\end{figure}
		\vfill
		\tiny Quelle: Beispielquelle für das Bild
		\begin{itemize}
			\item Streuexperimente als zentrale Methode
			\item Nachweis von Quark-Ladungen
		\end{itemize}
	\end{frame}
	
	% Section: Zusammenfassung
	\section{Zusammenfassung}
	\begin{frame}{Schlussfolgerungen}
		\frametitle{Zusammenfassung}
		\begin{itemize}
			\item Hadronen sind komplexe Strukturen aus Quarks
			\item Fortschritte in der Theorie und Experimente erforderlich
			\item Bedeutung des Quarkmodells für die moderne Physik
		\end{itemize}
	\end{frame}
	
	% Thank You Slide
	\begin{frame}{Vielen Dank!}
		\begin{center}
			\Huge Fragen?
		\end{center}
	\end{frame}
	
	% Bibliography Section
	\begin{frame}[allowframebreaks]{Literatur}
		\printbibliography\end{frame}
	
\end{document}
