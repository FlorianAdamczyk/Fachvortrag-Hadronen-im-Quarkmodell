\documentclass[aspectratio=169]{beamer} % 16:9 aspect ratio for modern screens

% Theme settings
\usetheme{metropolis} % Minimalist theme

% Packages
\usepackage[T1]{fontenc}   % Font encoding
\usepackage[ngerman]{babel} % German language
\usepackage[sfdefault]{FiraSans} % For FiraSans font
\usepackage[backend=biber, style=authoryear, sorting=ynt]{biblatex} % For bibliography
\usepackage{csquotes} % Recommended for biblatex with babel/polyglossia
\usepackage{textgreek} % Greek letters in text mode (aus references von Citavi)

% Bibliography settings
\addbibresource{references.bib} % Path to the bibliography file

% ...existing code...
\DeclareCiteCommand{\autortitel}
  {\usebibmacro{prenote}}
  {\usebibmacro{citeindex}%
   \printnames{labelname}%
   \setunit{\addcolon\space}% <- Hier wird das Trennzeichen ":" hinzugefügt
   \printfield{title}}
  {\multicitedelim}
  {\usebibmacro{postnote}}
% ...existing code...

% Title page settings
\title{Minimalistische Präsentation}
\author{Ihr Name}\date{\today}

\begin{document}

    \begin{frame}{Hauptteil}
        \begin{itemize}
            \item Das ist laut~\parencite{Zweig.1964b} wichtig.
            \item QUarks sind bunt~\fullcite{Wikipedia.Hadron}
            \item Bla Bla das geht laut~\autortitel{Wikipedia.Standardmodell} auch.
            \item Und das laut~\footcite{Zweig.1964} auch.
            \item wie funktioniert das hier:~\textcite{Wikipedia.Standardmodell}
            \item Und das:~\citetitle{Wikipedia.Hadron}
            \item wieso, funktioniert das~\autocite{GellMann.1964} auch?
            \item und in~\cite{Aaij.2015} sagen sie das.
            \item hier wirds spannend:~\footcitetext{Aaij.2015}
            \item Die Grundlagen des Quarkmodells wurden irgendwann entwickelt.~\parencite{Aaij.2019}.
        \end{itemize}
    \end{frame}

    % Bibliography
    \begin{frame}[allowframebreaks]{Literaturverzeichnis}
        \printbibliography\end{frame}

\end{document}
