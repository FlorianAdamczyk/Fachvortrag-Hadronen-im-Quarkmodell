\documentclass[aspectratio=169]{beamer} % 16:9 aspect ratio for modern screens

% Theme settings
\usetheme{metropolis} % Minimalist theme

% Packages
\usepackage[T1]{fontenc}   % Font encoding
\usepackage[ngerman]{babel} % German language
\usepackage[sfdefault]{FiraSans} % For FiraSans font
\usepackage[style=authoryear, sorting=ynt]{biblatex} % For bibliography
\usepackage{csquotes} % Recommended for biblatex with babel/polyglossia
\usepackage{textgreek} % Greek letters in text mode (aus references von Citavi)

% Bibliography settings
\addbibresource{references.bib} % Path to the bibliography file

% Title page settings
\title{Minimalistische Präsentation}
\author{Ihr Name}\date{\today}

\begin{document}

    \begin{frame}{Hauptteil}
        \begin{itemize}
            \item Das ist laut~\cite{Zweig.1964b} wichtig.
            \item Und das laut~\cite{Zweig.1964} auch.
            \item wie funktioniert das hier:~\cite{Wikipedia.Standardmodell}
            \item Und das:~\cite{Wikipedia.Hadron}
            \item wieso, funktioniert das~\cite{GellMann.1964} auch?
            \item und in~\cite{C.Amsler.2017} sagen sie das.
            \item hier wirds spannend:~\cite{Aaij.2015}
            % \item Die Grundlagen des Quarkmodells wurden irgendwann entwickelt.~\cite{Aaij.2019}.
        \end{itemize}
    \end{frame}

    % Bibliography
    \begin{frame}[allowframebreaks]{Literaturverzeichnis}
        \printbibliography\end{frame}

\end{document}
